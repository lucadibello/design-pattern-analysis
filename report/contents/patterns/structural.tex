\subsubsection{Adapter}

The tool detected the usage of the \textit{Adapter} design pattern in 6 different instances. To present a more interesting example, this section will focus on the \texttt{WeakInterner} final class. This class represents a \texttt{BoundedLocalCache<E,V>} object adapted to the \texttt{Interner} interface by wrapping it in a \texttt{WeakInterner} object that implements the methods specified by the \texttt{Interner} interface. The interface defines only one method \texttt{intern(E sample)}, which is a method common to all interned classes, that allows to add given an object to the \texttt{BoundedLocalCache} instance only if it is not already present in the cache. The \autoref{lst:weakinterner} showcase the general structure of the class.

\begin{lstlisting}[language=Java, caption={\texttt{WeakInterner} adapter structure}, label={lst:weakinterner}]
final class WeakInterner<E> implements Interner<E> {
  final BoundedLocalCache<E, Boolean> cache;
  ...
  @Override public E intern(E sample) {
    // Append to cache only if not present
  }
}
\end{lstlisting}

\noindent In summary, The \texttt{WeakInterner} class is a concrete implementation of the \texttt{Interner} interface that adapts the \texttt{BoundedLocalCache} object to the \texttt{Interner} interface in order to provide a more specific behavior.

\subsubsection{Decorator}

The tool detected the usage of the \textit{Decorator} design pattern in 3 different instances: \texttt{GuardedScheduler} (decorator of \texttt{Scheduler}), \texttt{BoundedWeigher} (decorator of \texttt{Weigher}) and \texttt{GuardedStatsCounter} (decorator of \texttt{StatsCounter}). Since this design pattern is implemented in a similar way in all found instances, only the \texttt{Weigher}-\texttt{BoundedWeigher} pair will be presented as it represents the most idiomatic usage of this design pattern among the found instances.

As we have seen in the previous section (refer to subsection \ref{sec:singleton}), the \texttt{Weigher} interface offers two static methods for the creation of specialized \texttt{Weigher} objects: \texttt{singletonWeigher} and \texttt{boundedWeigher}. The \texttt{BoundedWeigher} class is a decorates an existing \texttt{Weigher} object by adding a method \texttt{writeReplace()}, which is needed for serialization purposes (refer to the \href{https://docs.oracle.com/javase//7/docs/technotes/guides/serialization/examples/symbol/index3.html}{Oracle documentation}). To do so, the class uses the internally the \textit{delegate} object (\texttt{Weigher} instance passed by reference via constructor) to perform the actual work. The listing below showcases the main structure of the \texttt{BoundedWeigher} class:

\begin{lstlisting}[language=Java, caption={\texttt{BoundedWeigher} final class decorating a \texttt{Weigher} instance}, captionpos=b]
final class BoundedWeigher<K, V> implements Weigher<K, V>, Serializable {
  @SuppressWarnings("serial")
  final Weigher<? super K, ? super V> delegate;

  BoundedWeigher(Weigher<? super K, ? super V> delegate) {
    ...
  }
}

\end{lstlisting}

\noindent As \texttt{BounderWeigher} implements the \texttt{Weigher} interface, it must implement the same methods as the \textit{delegate} object passed to the constructor. By encapsulating the \textit{delegate} object function calls, the \texttt{BoundedWeigher} class is able to extend the default behavior of the object while maintaining the same interface. This is the essence of the \textit{Decorator} pattern.

