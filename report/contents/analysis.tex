\section{Analysis of results}

After running the design pattern analysis tool on the entire \texttt{com.github.benmanes.caffeine.cache} package, the tool was able to find the usage of seven design patterns. In this report we will analyze the most idiomatic usages of the design patterns found in the \textit{Caffeine} library.

The following table summarizes the findings of the design pattern analysis tool:

\begin{itemize}
	\item \textbf{Factory Method}: Detected in 5 different classes, with a total of 7 factory methods identified. In \textit{Caffeine} is is used to create particular configuration of data structures, such as synchronous or asynchronous caches, or to create a certain type of cache node.

	\item \textbf{Singleton}: Found in 9 different classes, with a total of 9 singleton instances. Due to the versatility of the pattern, it has been used extensively in the library for various reasons. The analyzer unfortunately was not able to detect usage of this pattern in the \texttt{NodeFactory} interface (refer to \autoref{par:nodefactory} for more details)

	\item \textbf{Adapter}: Detected in 6 different classes, with 6 adapter classes detected. This pattern allows the library to use internally the interface of a passed object to perform certain operations, while exposing a different interface to the client. For instance, \texttt{BoundedPolicy} adapts the functionality of a \texttt{BoundedLocalCache} object to perform certain operations on the cache, while exposing a \texttt{Policy} interface to the client.

	\item \textbf{Decorator}: Identified in 3 different classes. \textit{Caffeine} library use this pattern to extend the functionality certain base classes. For example, the class \texttt{GuardedScheduler} decorates a \texttt{Scheduler} object to add additional functionality to the scheduler, while maintaining the original interface.

	\item \textbf{State}: This design pattern has been identified in 8 classes. This design pattern allows to encapsulate different behavioral states of an object in separate classes, allowing the object to change its behavior at runtime. Unfortunately, the scanner produces many false positives for this pattern, as it is difficult to distinguish between a state pattern and a simple delegation of functionality to another object. Please refer to \autoref{sec:false-positives} for more details.

	\item \textbf{Bridge}: Found in 4 classes, this design pattern allows to decouple the object abstraction from its implementation, allowing both to change independently. Unfortunately, the scanner also in this case produced many false-positives, please refer to \autoref{sec:false-positives} for more details.

	\item \textbf{Template Method}: Detected the use of this design pattern in 7 different classes. In most of the cases the library uses the template method in classes reguarding data structure definitions, in order to separate the logic of the data structure from the general logic, allowing subclasses to redefine certain behaviors. Unfortunately even in this case the scanner made some mistakes, refer to the \autoref{sec:false-positives} for more details.

\end{itemize}

\noindent Due to the size of the library, and the amount of design patterns detected, the report will only analyze the most interesing usages of the design patterns found. For each design pattern, the report will provide details on how the pattern is implemented and what value it brings to the architecture of the library.

\textit{Note:} due to the limited amount of pages, it is impossible to analyze in detail every detected design pattern and its usage inside the library. For this reason, certain design patterns, where the usage is not interesting or where the scanner produced multiple false-positives, will not be analyzed in the following sections.

\subsection{Creational Design Patterns}

\subsubsection{Factory Method}
\label{par:nodefactory}

The tool detected the usage of the \textit{Factory Method} design pattern in 5 different instances. This pattern is used to encapsulate the creation of objects, allowing subclasses to return different instances of the same type using polymorphism. The pattern is used in the \textit{Caffeine} library to create particular configurations of data structures, such as synchronous or asynchronous caches, or to create particular types of cache node objects.

For example, the \texttt{NodeFactory} interface represents a particular implementation of the \textit{Factory Method} pattern. It defines a two static constructor methods \texttt{newFactory} to create new \texttt{NodeFactory} instances based on a specified class name. The method signatures are defined in \autoref{lst:nodefactory}.

\begin{lstlisting}[language=Java, caption={\texttt{NodeFactory} interface static constructor methods}, captionpos=b, label={lst:nodefactory}]
static NodeFactory<Object, Object> newFactory(String className)
static <K, V> NodeFactory<K, V> newFactory(Caffeine<K, V> builder, boolean isAsync)
\end{lstlisting}

\noindent These two static constructors leverage the \textit{Singleton} design pattern in order to provide a single instance of the \texttt{NodeFactory} per class name. This behavior is achieved by using a static \texttt{ConcurrentMap} to store the instances of the \texttt{NodeFactory} class linked to the class name (passed explicitly via parameter, or embedded in the \texttt{builder} object). This way, the method can simply return the instance if it is already present in the map, or create a new one if it does not exist yet. The code responsible for this behavior is presented in \autoref{lst:nodefactorysingleton}.

It is important to highlight that the design pattern scanner, probably due to the complexity of the implementation, did not detect the usage of the \textit{Singleton} design pattern in the \texttt{NodeFactory} interface. This is a clear example of the limitations of the tool, as the pattern is clearly implemented in the codebase.

\begin{lstlisting}[language=Java, caption={\texttt{NodeFactory} \textit{Singleton} design pattern implementation using a static \texttt{ConcurrentMap} instance}, captionpos=b, label={lst:nodefactorysingleton}]
    var factory = FACTORIES.get(className);
    if (factory == null) {
      factory = FACTORIES.computeIfAbsent(
        className, NodeFactory::newFactory
      );
    }
    return (NodeFactory<K, V>) factory;
\end{lstlisting}

\noindent Furthermore, the \texttt{NodeFactory} interface also defines two abstract factory methods to create new cache nodes (\texttt{Node} objects) of different types. The method signatures are defined in \autoref{lst:nodefactorymethods}.

\begin{lstlisting}[language=Java, caption={\texttt{NodeFactory} abstract factory methods to create \texttt{Node<K,V>} objects}, captionpos=b, label={lst:nodefactorymethods}]
Node<K, V> newNode(K key, ReferenceQueue<K> ref, V value,
  ReferenceQueue<V> valueReferenceQueue, int weight, long now);
Node<K, V> newNode(Object keyReference, V value,
  ReferenceQueue<V> valueReferenceQueue, int weight, long now);
\end{lstlisting}

\noindent Since both methods are \texttt{abstract}, each concrete implementation of the \texttt{NodeFactory} interface must define its own implementation of these methods, allowing to return different instances of the \texttt{Node} interface. This is a clear example of the \textit{Factory Method} pattern usage.

As example, the \texttt{BoundedLocalCache} cache internally use a \texttt{NodeFactory} factory to create new cache nodes. This allows to easily switch between different implementations of the \texttt{Node} interface, depending on the cache type.

\subsubsection{Singleton}
\label{sec:singleton}

The \textit{Singleton} pattern is used to ensure that a particular class has only one instance. To address this requirement, the class usually defines a private constructor (to hide the constructor method from clients), and a static method to serve as the new entry point to the instance. This method will try to lazily create the instance, if it does not exist yet, and return it.

The tool detected the usage of the \textit{Singleton} pattern in the \textit{Caffeine} library in 9 different instances. Since in all detected instances the pattern is implemented in similar ways, I will present only one instance where the pattern is used in a more interesting way.

The \textit{Weigher} interface is used to calculate the weight of a given cache node. In order to prevent the creation of multiple \texttt{Weigher<K, V>} instances, the \textit{Caffeine} library implements the \textit{Singleton} pattern by leveraging a very common Java trick involving an \textit{enum} type, which in this case is named \texttt{SingletonWeighter}. The enum definition is available in \autoref{lst:singletonweigher}.

\begin{lstlisting}[language=Java, caption={\texttt{SingletonWeigher} enum definition}, captionpos=b, label={lst:singletonweigher}]
enum SingletonWeigher implements Weigher<Object, Object> {
  INSTANCE;
  // dummy weigh method
  @Override public int weigh(Object key, Object value) {
    return 1;
  }
}
\end{lstlisting}

\noindent The \textit{SingletonWeighter} enum implements the \textit{Weigher} interface, and defines a single entry named \texttt{INSTANCE}, which is of type \texttt{SingletonWeigher}. Since the \texttt{SingletonWeighter} enum implements the \texttt{Weigher} interface, the \texttt{INSTANCE} entry is also of type \texttt{Weigher<Object, Object>}, effectively making it a singleton instance of the \texttt{Weigher} interface. This is only possible due to the intrinsic nature of \textit{enum} types, where each entry is guaranteed to be unique and only instantiated once.

In order to retrieve the singleton instance in a more concise way, the \texttt{Weigher} interface offers a static \texttt{singletonWeigher()} utility method that simply reads the \texttt{INSTANCE} entry from the \texttt{SingletonWeigher} enum and performs an unchecked cast to the \texttt{Weigher<K, V>} type, effectively returning the singleton instance:

\begin{lstlisting}[language=Java, caption={SingletonWeigher shorthand method to return the singleton instance using an unchecked cast}, captionpos=b, label={lst:singletonweigher}]
static <K, V> Weigher<K, V> singletonWeigher() {
  @SuppressWarnings("unchecked")
  var instance = (Weigher<K, V>) SingletonWeigher.INSTANCE;
  return instance;
}
\end{lstlisting}

\noindent Additonally, this interface exhibit also some characteristics of the \textit{Factory Method} pattern, offering custom methods that encapsulate the creation of different kinds of \textit{Weigher} objects. The method signatures are defined in \autoref{lst:weigherfactorymethods}.

\begin{lstlisting}[language=Java, caption={\texttt{Weigher} interface utility methods to create different kinds of \texttt{Weigher<K,V>}}, captionpos=b, label={lst:weigherfactorymethods}]
static <K, V> Weigher<K, V> singletonWeigher()
static <K, V> Weigher<K, V> boundedWeigher(Weigher<K, V> w)
\end{lstlisting}

\noindent However, in order to be a correct \textit{Factory Method} design pattern implementation, subclasses should be able to implement the \textit{Weigher} interface and define their own custom implementations of the factory methods presented above, allowing them to leverage polymorphism in order to return different instances of the \textit{Weigher} interface. This is not possible as both methods are \textit{static}, thus not allowing subclasses to alter the return type of the methods via inheritance as required by the design pattern.



\subsection{Structural Design Patterns}

\subsubsection{Adapter}

The tool detected the usage of the \textit{Adapter} design pattern in 6 different instances. To present a more interesting example, I will focus on the \texttt{BoundedLocalCache} abstract class. This abstract class represents a bounded cache implements the \texttt{LocalCache} interface to offer a consistent API to the client. The \texttt{BoundedLocalCache} class is defined as follows:

\subsubsection{Decorator}

The tool detected the usage of the \textit{Decorator} design pattern in 3 different instances: \texttt{GuardedScheduler} (decorator of \texttt{Scheduler}), \texttt{BoundedWeigher} (decorator of \texttt{Weigher}) and \texttt{GuardedStatsCounter} (decorator of \texttt{StatsCounter}). Since this design pattern is implemented in a similar way in all found instances, I will only present the \texttt{Weigher}-\texttt{BoundedWeigher} pair as an example as it is the most idiomatic across the codebase.

As we have seen in the previous section (refer to subsection \ref{sec:singleton}), the \texttt{Weigher} interface offers two static methods for the creation of \texttt{Weigher} objects: \texttt{singletonWeigher} and \texttt{boundedWeigher}. The \texttt{BoundedWeigher} class is a concrete implementation of the \texttt{Weigher} interface that decorates another \texttt{Weigher} object by adding an additional method \texttt{writeReplace()} to the object (needed for serialization, refer to the \href{https://docs.oracle.com/javase//7/docs/technotes/guides/serialization/examples/symbol/index3.html}{Oracle documentation}). The class uses internally the \textit{delegate} object to perform the actual work. The listing below showcases the main structure of the \texttt{BoundedWeigher} class:

\begin{lstlisting}[language=Java]
final class BoundedWeigher<K, V> implements Weigher<K, V>, Serializable {
  @SuppressWarnings("serial")
  final Weigher<? super K, ? super V> delegate;

  BoundedWeigher(Weigher<? super K, ? super V> delegate) {
    ...
  }
}

\end{lstlisting}

\noindent As \texttt{BounderWeigher} implements the \texttt{Weigher} interface, it must implement the same methods as the \textit{delegate} object passed to the constructor. By encapsulating the \textit{delegate} object function calls, the \texttt{BoundedWeigher} class is able to extend the default behavior of the object while maintaining the same interface. This is the essence of the \textit{Decorator} pattern.



\subsection{False Positives: State, Bridge and Template Method design patterns}
\label{sec:false-positives}

The scanner produces many false positives for the following \textit{State}, \textit{Bridge} and \textit{Template Method} design patterns.

For example, the scanner detected the usage of the \textit{State} design pattern inside the \texttt{BoundedLocalCache}. This class implements a task-based approach to alter the state of specific nodes inside the cache using a \textit{page replacement policy}. To update the page replacement policy, the class defines three types of task: \texttt{AddTask}, \texttt{UpdateTask} and \texttt{RemovalTask}. These task have a common structure, offering a single constructor accepting a \texttt{Node<K, V>} object, and implementing the \texttt{Runnable} interface. The \texttt{run} method of each task defines the behavior of that specific task.

The scanner detected the usage of the \textit{State} design pattern since it detected that each task is modeling a specific state of the node object passed as reference. On the other hand, after further inspection, this structure represents perfectly the \textit{Command} design pattern: the task represents a particular request to alter the state of the node, and the \texttt{run} method is the actual command that will be executed on the node. Furthermore, after the creation of a specific task, the program appends it to a global buffer named \texttt{writeBuffer}. This buffer will then be used to pull tasks and execute them at specific times (i.e. after a write operation). This represents the essence of the \textit{Command} design pattern.

The \textit{Bridge} design pattern on the other hand was detected in many places across the codebase, where it found

I was expecting the scanner to produce false positives, especially for behavioral design patterns, as it is difficult to distinguish between a simple delegation of functionality to another object and the actual usage of a design pattern. In this library, where the class hierarchy is very complex and well-structured, I was expecting the scanner to produce many false positives. Please refer to the \autoref{sec:analysis} for more details about the false-positives I found

\subsection{Expected Patterns}

The \textit{Caffeine} library is an important open-source project, that has been around for nearly 10 years (first commit on the \displaydate{caffeinerelease}). For this reasons, I was expecting to find a well-structured library with a clear separation of concerns and a good usage of design patterns. From the analysis made above, we can confirm all of the assupmtions. Unfortunately, I was also expecting the use of certain design patterns that were not found by the scanner, but that are commonly used for this type of utility library:

\begin{itemize}
	\item Observer: in order to notify clients of changes in the cache (i.e. evictions, insertions, etc.), the library should implement the observer pattern.
	\item Command: the library should use the command pattern to encapsulate requests as objects, allowing to parameterize clients with queues, requests and operations.
\end{itemize}

\noindent Both of the patterns were not detected by the scanner, but after further analysis of false-positives produced by the scanner, was possible to identify the \textit{Command} design pattern in the \texttt{BoundedLocalCache} class (refer to \autoref{sec:false-positives}).

On the other hand, by a manual inspection of the codebase (searching for common keywords used commonly in the observer pattern, i.e. \textit{listener}, \textit{observer}) it was not possible to find any usage of the observer pattern in the actual codebase, but rather the use of this design pattern when using some dependencies of the library.

